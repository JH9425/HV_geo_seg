\section{Concluding Remarks}\label{sec:conclusion}
Understanding the margins of geographic market segmentation is central to discerning whether there is still scope for continued integration efforts across European markets and if so which margin should be targeted by policymakers. Recent studies have reiterated the continued existence of large price differences and differences in trade shares across European countries. However, only studying LOP deviations is potentially a too narrow view and solely looking at regional variation in trade shares risks convoluting taste differences with limits to geographic market integration. 

This paper combines household-level scanner data across European regions with a structural model of preferences to detect geographic market segmentation in terms of regional cost-of-living differences. By relying on an intuitive decomposition of the cost-of-living differences, we decompose cost-of-living differences into (1) LOP deviations, (2) pure taste differences and (3) choice set differences. In this way, we detect geographic market segmentation by considering whether LOP deviations and choice set differences are higher for international region pairs relative to intranational region pairs with similar geographic differences.

We find that geographic market segmentation is still large among European countries as LOP deviations and choice set differences lead to cost-of-living differences that are more than three times larger for international region pairs relative to intranational pairs. While LOP deviations contribute to these larger cost-of-living differences, they are of second-order importance. Differences in choice sets are responsible for 95\% of geographic market segmentation in Europe. 

Importantly, we estimate that roughly 60\% of the increased cost-of-living differences across international pairs are related to differences in consumer taste. This implies two things. First, large cost-of-living differences across European countries will likely remain even after the sources of geographic market segmentation have been removed. As the same implication holds for trade shares, border effects for trade shares should be interpreted with caution. 

Our results further imply that to reduce geographic market segmentation, policymakers should likely focus on stimulating cross-country entry of firms and product varieties to reduce choice differences. Unfortunately, while being very rich, our data does not allow us to relate cost-of-living differences stemming from choice set differences to specific policies. Still, our approach can rule out certain reasons for their continued existence. First, all countries we study are part of the European customs and currency union. This ensures that the remaining level of geographic market segmentation is not due to conventional barriers to trade such as tariffs, quotas or currency conversion costs. Second, the estimated level of geographic market segmentation is unlikely to be driven by transport costs as we compare quite small and homogeneous geographic regions. This leads us to speculate that certain informal barriers are still segmenting European markets, but we leave it to future work to help uncover those. 