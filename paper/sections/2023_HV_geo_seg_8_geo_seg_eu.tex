\section{Geographic Market Segmentation in Europe}\label{sec:border_effects_eu}
There are two reasons why a simple comparison of conditional means in cost-of-living differences between international and intranational region pairs will lead to an overestimation of the level of geographic market segmentation. First, Table \ref{tab: var_decomp_cle} illustrated that the conditional distribution of pure taste differences for international pairs is shifted upwards relative to the distribution for intranational pairs. As pure taste differences are traditionally considered to be outside of market integration policies, we will filter out pure taste differences when assessing the presence of geographic market segmentation. Second, geographic differences between international pairs are larger compared to intranational pairs. As geographic differences are likely positively correlated with the cost of physically moving goods to their final destination markets, simply comparing intranational and international region pairs will lead to an overestimation of cost-of-living differences due to geographic market segmentation. For this reason, we first elaborate on an empirical design that controls for these geographic differences across region pairs. 

\subsection{Empirical design}
To condition on geographic differences across regions, we introduce a potential outcomes framework that draws heavily on the ideas developed in \citet{Santamaria2021}. 
\paragraph{Potential outcomes framework}    We consider the following potential outcomes framework 
\begin{linenomath*}
    \begin{equation*}
        \mathbb{V}_{ll'} = 
            \begin{cases}
                & \mathbb{V}_{ll'}(0) \qquad \text{if} \quad B_{ll'} = 0 \\
                & \mathbb{V}_{ll'}(1) \qquad \text{if} \quad B_{ll'} = 1
            \end{cases}
    \end{equation*} 
\end{linenomath*}
\noindent where $\mathbb{V}_{ll'}$ is the variance of cost-of-living differences between region $l$ and region $l'$. Define $\mathbb{V}_{ll'}(0)$ as the value of the variance that would have materialized if region $l$ and region $l'$ were intranational region pairs and $\mathbb{V}_{ll'}(1)$ as the variance if they were an international region pair. We define the effect of being separated by a national border as:
\begin{linenomath*}
    \begin{equation}\label{eq:estimand}
        \tau \equiv \mathbb{E}\left[\mathbb{V}(1) - \mathbb{V}(0)|B_{ll'} = 1\right]
    \end{equation}
\end{linenomath*}
\noindent  By defining the effect in this way, the border effects measure the increase in the variance of cost-of-living differences for international pairs that are currently being by a national border compared to when they would only be separated by a regional border.\footnote{In terms of the causal inference vocabulary, our estimand corresponds to an average treatment effect on the treated. We believe that considering how cost-of-living differences would have looked if there were no national border seems like a more sensible approach than an average treatment effect or an average treatment effect on the untreated. This is because in this case, we would have to impose national border assignments on certain intranational region pairs and have to take a stance on what it would do to other adjacent intranational pairs. Instead, taking the current national border assignment as given and estimating the impact of those existing national borders seems like a more reasonable thought experiment.}\footnote{In addition, the bias due to treatment effect heterogeneity as mentioned by \citet{Gorodnichenko2009} will not affect our estimates. This is because by defining the estimand as an average treatment effect on the treated (ATT) and not as an average treatment effect (ATE) differences in the treatment effect on the treated and on the untreated (ATU) do not enter the estimand anymore.} Consistent estimation of the border effect requires individualistic, probabilistic, unconfounded assignment and compliance with the assignment. We discuss the plausibility of these assumptions now. 

First, consistent estimation requires the SUTVA assumption that requires that (1) separating a region pair by a national border does not affect the potential outcomes of other region pairs and (2) that separating region pairs by a national border cannot be executed in multiple ways. The first part of this assumption requires that the national border assignment is individualistic and thus rules out that separating one region pair affects the potential outcomes of other region pairs. In general, spatial spill-overs might occur if changes in national border assignment changed the size of the relevant market. However, the size of these spill-overs might be small given the disaggregated spatial level at which we define the treatment variable. There are 3,403 region pairs we consider. If we were to allocate a Belgian region to the Netherlands, there would be 9 additional borders with Belgium and 12 fewer borders with the Netherlands. This amounts to a 0.6\% change in the number of treated units. While this number is not zero, it is small and it seems reasonable to assume that the change in the aggregate economic determinants of the market segmentation would be small. The second part requires that separating regional pairs by a national border does not vary across region pairs. Given the clear separation of national and supranational regulatory, legal and tax responsibilities between member states and the European institutions and the lack of additional bilateral agreements across member states, it seems plausible to assume that what matters up to a first-order is the separation by a national border and not the specific bilateral border.\footnote{While the border assignment is assumed to be equal across region pairs, this assumption does not rule out the possibility that border effects are heterogeneous across region pairs.}

Second, border assignment needs to be probabilistic. In other words, every region pair needs to have a probability of being separated by a border strictly different from zero and one. This assumption is likely to hold in our setting given that both contiguous regions and very spatially separated region pairs are separated by region and national borders in the data.

Third, to rule out geographic differences as a source of cost-of-living differences, we require that national border assignment was not chosen with the potential outcomes or the potential degree of market segmentation in mind. The results from section \ref{sec:reduced_form} show that conditional on a national border distance affects price dispersion and choice set differences. Hence, geographic differences are correlated with the degree of market segmentation. If national border assignment is also determined by geographic differences, a simple comparison of international and intranational regions, along the lines of \citet{Engel1996}, will not yield a reliable estimate (see \citet{Gorodnichenko2009}). To estimate a border effect that is not confounded by geographic characteristics, we follow the approach taken in \citet{Santamaria2021} and estimate the border effect by conditioning on a set of geographic covariates that are correlated with national border assignment:
\begin{linenomath*}
    \begin{equation*}
        \{\mathbb{V}_{ll'}(0),\mathbb{V}_{ll'}(1)\} \perp \!\!\! \perp B_{ll'} | \boldsymbol{X}_{ll'} 
    \end{equation*}
\end{linenomath*}
where $\boldsymbol{X}_{ll'}$ includes great-circle distance, remoteness, shared river basins and altitude differences between region $l$ and $l^{w'}$. This approach has merit for two reasons. First, geography can be considered as a pre-treatment variable that determines border assignment. It is well-known that mountainous areas and rivers have shielded nations from invasions (e.g. \citet{Nunn2012}) and more distant and larger populations are more difficult to govern \citet{Alesina1997}. Second, even if there are many potential explanations for border effects and we are unable to isolate one particularly important source, this approach eliminates geographic frictions, such as physical distance, as a potential explanation for the estimated border effects. 

Finally, in the presence of LOP deviations or choice set differences, consumers may engage in cross-border shopping. If so, this would result in non-compliance with national border assignments. To understand the importance of such non-compliance, we use the Belgian data for which the variable containing the store name indicates whether the store is located in Belgium or in one of the neighboring countries.\footnote{To be precise, the variable indicates whether the store is a Belgian, French, German, Dutch store or whether the store is a foreign one.} As previous studies have shown that Belgium appears to have higher consumer prices for the products we study (e.g. \citet{Beck2020}) and Belgium is well-connected to its neighboring countries, cross-border shopping would manifest itself, especially in Belgium. While there is some cross-border shopping, Table \ref{tab: app_border_eff_spectest_cbshopping_overall} shows that over 97\% of expenditure by Belgian households is made in stores located in Belgium. Since this table pools across all Belgian households, Figure \ref{fig: app_border_eff_spectest_cbshopping_distance} focuses on the importance of cross-border shopping close to the Belgian-French and the Belgian-Dutch border.\footnote{Cross-border shopping does not seem to be relevant for the Belgian-German border at all.} We plot the propensity that households have engaged at least once in cross-border shopping and the overall expenditure share of cross-border shopping as a function of the distance to the border. While the propensity that households engage at least once in cross-border shopping over the sample period tends to increase when we approach a border, the overall expenditure share on cross-border transactions in very close proximity to the border remains low at a little over 5\% and 10\% for the French and Dutch borders respectively. We conclude that non-compliance with national borders is not a major issue in our data. 

\paragraph{Propensity score and subclassification} Table \ref{tab: app_prop_score_sum_stats} shows the conditional distributions of the geographic variables across intranational and international pairs. The t-tests and overlap tests highlight that there exist important imbalances between these conditional distributions. To ensure that geographic differences are not responsible for the estimated border effects, we implement a two-step approach that ensures that we only use variation across region pairs equally likely to be separated by a national border. In the first step, we estimate the probability that two regions are separated by a national border based on geographic determinants. In the second step, we trim the sample and construct groups of intranational and international pairs with very similar estimated probabilities of being separated by a national border. We estimate the propensity score that two regions are separated by a national border by assuming that the probability of being separated by a national border takes the following logistic form: 
\begin{linenomath*}
    \begin{equation}\label{eq:prop_score}
        \text{P}(B_{ll'} = 1|\boldsymbol{X}_{ll^{'}}) 
            = \frac{e^{ \beta_0 + \boldsymbol{X}_{ll'}\boldsymbol{\beta}_1 + \boldsymbol{X}_{ll'}\cdot\boldsymbol{X}_{ll'}\boldsymbol{\beta}_2 + \varepsilon_{ll}}}
                   {1 + e^{\beta_0 + \boldsymbol{X}_{ll'}\boldsymbol{\beta}_1 + \boldsymbol{X}_{ll'}\cdot\boldsymbol{X}_{ll'}\boldsymbol{\beta}_2 + \varepsilon_{ll}}}
    \end{equation}
\end{linenomath*}
\noindent where $\boldsymbol{X}_{ll'} =$ \{Distance, Remoteness, Altitude difference, Shared River basin\} is the set of geographic determinants. We consider a second-order polynomial in the geographic variables as the set of possible explanatory variables. We follow \citet{Imbens2015} by iteratively selecting from this set the variables that improve the explanatory power of the model the most and stop when we cannot improve the predictive power of the model meaningfully.\footnote{In practice, we start from a first-order polynomial in the geographic variables and check which second-order term adds the most explanatory power to the model by comparing the restricted model with an augmented model that includes one second-order term. Hereafter, we add this variable to the restricted and search for the next variable that improves the explanatory power the most. We stop when we cannot further increase the explanatory power by a given threshold. This threshold is a likelihood ratio test score of 2.71 as recommended by \citet{Imbens2015}.} Table \ref{tab: app_prop_score_model_selection} presents the outcome of this iterative selection procedure and shows how we arrive at the final set of covariates. Columns (1) and (2) of Table \ref{tab: app_prop_score_estimation} present the estimation results after estimating \ref{eq:prop_score} with maximum likelihood. Adding the selected second order-terms to the restricted first-order model leads to an increase in the pseudo-$R^2$ from 0.2 to 0.28. 

Figure \ref{fig: app_prop_score_full} presents the conditional distribution of estimated propensity scores across international and intranational regions. While there is considerable overlap in the conditional distributions, there is much less overlap in the tails. This is problematic for two reasons. First, sampling variance increases when we were to compare a small number of intranational pairs with a large set of international pairs as it puts a very large weight on a few control units. To limit the contribution of region pairs in the tail ends of the distribution, we implement the algorithm developed in \citet{Crump2009} and exclude observations with an estimated propensity score outside of a critical interval of $[\alpha,1-\alpha]$.\footnote{The idea behind the algorithm is to search for the cut-off value that balances the increase in variance by reducing the sample size and the decrease in variance because of excluding observations in the tail of the distribution. Figure \ref{fig: app_prop_score_trimming_algo} shows the output of the grid search algorithm.} The algorithm reaches its critical point at $\alpha = 0.093$ such that we remove a little over 20\% of the region pairs (see Table \ref{tab: app_prop_score_estimation}). Figure \ref{fig: app_prop_score_trimmed} and Figure \ref{fig: app_prop_score_trimmed_re} show the improved overlap in the conditional distributions of the estimated propensity scores once after trimming the sample and once after trimming the sample and re-estimating the propensity scores after trimming. Second, comparing intranational pairs with low probabilities of being separated by a national border and international pairs with a high probability of being separated requires substantial extrapolation. Even after trimming the sample appropriately, we still need to ensure that we only compare intranational and international pairs with roughly equal probabilities of being separated by a national border. To this end, we subdivide the intranational and international pairs into groups in which the normalized difference in the estimated propensity between intranational and supranational is below a threshold value. We implement the algorithm developed in \citet{Imbens2015} and subdivide the region pairs into 7 groups. Table \ref{tab: app_prop_score_blocks} provides an overview of the number of international and shows that within each block the standardized differences in the estimated propensity score are small between international and intranational pairs. Given these groups, we estimate the border effect through the following estimator: 

\begin{linenomath*}
    \begin{equation}\label{eq:att_subclass}
        \hat{\tau} = \sum_{j = 1}^{J = 7} \frac{N_{\text{int}}(j)}{N_{\text{int}}(j) + N_{\text{nat}}(j)}\hat{\tau}(j), \qquad 
        \hat{\tau}(j) = 
            \argmin_{\tau(j)}
            \sum_{ll'}\mathbb{1}
            \left(ll' \in \mathcal{G}(j)\right)\cdot
            \left(\mathbb{V}_{ll'} - \beta_0 - \tau B_{ll'} - \boldsymbol{X}_{ll'}\boldsymbol{\beta}_1\right)^2
    \end{equation}
\end{linenomath*}

\noindent where $\mathcal{G}(j)$ indicates the set of region pairs included in group $j$ and $N_{\text{int}}$ and $N_{\text{int}}$ are the number of national and international region pairs in group $j$.\footnote{This weighting scheme ensures that we obtain the Average Treatment Effect on the Treated.} We include the geographic variables to improve precision and to remove the potential bias from any remaining unbalance in conditional distributions of the geographic determinants. 

\subsection{Cost-of-living Differences for Comparable Regions}
Before discussing the results in which we detect the presence of geographic market segmentation, we provide an estimate of the overall cost-of-living differences by comparing intranational and international region pairs with similar geographic differences.

\paragraph{Basline results} Table \ref{tab: border_effects_cle} presents the border effect estimates of cost-of-living differences at the baseline elasticities of substitution estimates for different specifications. We report the average variance of log cost-of-living differences for intranational pairs in the row indicated by $\mathbb{V}\left(\cdot|l,l^{''} \in \mathcal{L}_{\text{intra}} \right)$ for each of the specifications. Column (1) shows the results when we estimate the border effect as we did in section \ref{sec:reduced_form}. We regress the variance in cost-of-living differences across region pairs and years on a national border dummy and include time fixed effects. Consistent with Table \ref{tab: var_decomp_cle}, the variance of cost-of-living differences is on average over 8 times larger for international pairs relative to intranational pairs. When we control for geographic differences in column (2), cost-of-living differences are more than 7 times larger. However, these estimates potentially compare very geographically different intranational and international pairs and could overestimate the border effect. To this end, we first trim the sample by the critical value established in the previous and keep only region pairs with a propensity score in the range of $[0.093,0.907]$. In this case, the cost-of-living differences fall in magnitude for both international and intranational region pairs, but the difference in cost-of-living differences is still close to over 7 times larger for international pairs. Second, in column (4) we apply the subclassification estimator from equation \ref{eq:att_subclass} on the trimmed sample. In this way, we estimate the border effect by only comparing international and intranational pairs with very similar geographic differences. Compared to column (3), the difference in cost-of-living differences is essentially unchanged and the variance in cost-of-living differences remains 7 times larger for international pairs. Comparing the estimate in column (4) to the one in column (1), geographic differences account for around 12\% of the estimated cost-of-living differences across the countries in our sample.  

%\paragraph{Robustness}  We check the robustness of the results regarding the elasticities of substitution used and the inclusion of outliers. First, while the baseline product variety level elasticities of substitution are in line with some recent papers, they are lower in absolute value than the estimates reported in \citet{Hottman2016}. As the product variety level expenditure share and choice set differences fall with the elasticity of substitution, we check the robustness of our results when we consider higher values for the product-variety level elasticities. We re-estimate the specification used in column (4) by shifting the distribution of estimated product-variety level elasticities by 1 in column (5) and by 2 in column (6).\footnote{In this way, we respect the absolute level of heterogeneity in the estimated elasticities, but shift the location of the distribution.}. As expected the absolute size of the border effect falls when the elasticities of substitution used to compute the two product-variety level components rise. However, because the estimated cost-of-living differences for the set of intranational pairs also fall, the ratio of the two variances remains relatively stable at 7.1 and 7 in columns (5) and (6) respectively. Second, Figures \ref{fig: redform_choice} in section \ref{sec:reduced_form} and Figures \ref{fig: struc_est_rel_share} showed that the distributions of relative product variety- and firm-level expenditure shares on common product varieties and firms are very dispersed. To ensure that the border effect results are not driven by outliers, we re-estimate the specification in column (4) for different locations of the product variety level elasticities of substitution and for different levels of winsorizing applied to the distributions of the structural components before computing the variances. Table \ref{tab: app_border_effects_cle_sens} shows that the border effects vary between $0.9288$, the estimate in column (4), to $0.694$, the estimate when we winsorize all distributions at 10\%, both evaluated at the baseline elasticities. When we shift the distribution of product variety level elasticities of substitution by 2 the non-winsorized estimate is $0.619$, the estimate reported in column (6), and $0.556$ which is the estimate when we winsorize all distributions at 10\%. 

\paragraph{Robustness}  We check the robustness of the results regarding the elasticities of substitution used. While the baseline product variety level elasticities of substitution are in line with some recent papers, they are lower in absolute value than the estimates reported in \citet{Hottman2016}. As the product variety level expenditure share and choice set differences fall with the elasticity of substitution, we check the robustness of our results when we consider higher values for the product-variety level elasticities. We re-estimate the specification used in column (4) by shifting the distribution of estimated product-variety level elasticities by 1 in column (5) and by 2 in column (6).\footnote{In this way, we respect the absolute level of heterogeneity in the estimated elasticities, but shift the location of the distribution.}. As expected the absolute size of the border effect falls when the elasticities of substitution used to compute the two product-variety level components rise. However, because the estimated cost-of-living differences for the set of intranational pairs also fall, the ratio of the two variances remains relatively stable at 6.8 and 6.7 in columns (5) and (6) respectively.

\begin{table}[H]
        \centering
        \caption{Border effect: Cost-of-living}
        \label{tab: border_effects_cle}
        \begin{spacing}{1.1}
            \scalebox{0.9}{
            \begin{tabular}{lcccccc} \toprule 
                $\mathbb{\hat{V}}\left[p_{l^{'},t}-p_{l,t}\right]$ & (1) & (2) & (3) & (4) & (5) & (6) \\ 
                \midrule
                \myinput tables/estimation/P1_3_4_3_NCES_cl.tex \bottomrule \\
        \end{tabular}}
    \end{spacing}
        \parbox{\textwidth}{
        \begin{spacing}{1} 
            {\footnotesize 
            \textit{Notes}: This table presents the border effect estimates for cost-of-living differences across international and intranational pairs. Cost-of-living differences are computed based on Equation \ref{eq:cle_decomp}. Hereafter, we take a logarithmic transformation of these level differences and compute the variance across product categories within region pair-year cells. Column (1) estimates the border effect by regressing the variance of log cost-of-living differences on a border dummy and year fixed effects. Column (2) adds geographic controls. Column (3) takes the regression from column (2) and estimates the border effect on the trimmed sample (based on the admissible propensity score range). Column (4) applies the subclassification estimator on the trimmed sample. Columns (5) and (6) re-compute cost-of-living differences by shifting the product variety level elasticity of substitution distribution by 1 and 2 respectively. Hereafter, border effects are estimated using the subclassification estimator applied to the trimmed sample. Beneath each specification, we disclose the simple average of the variance of log cost-of-living differences for the intranational region pairs included in the estimation sample. We compute cluster standard errors at the region pairs and present them in brackets below the coefficient estimates. Reported significance levels are at the $p<0.1^{*}$,$p<0.05^{**}$ and $p<0.01^{***}$ levels.} 
    \end{spacing}}
 \end{table}

\subsection{Detecting Geographic Market Segmentation}  
The final step to detect the presence of geographic market segmentation is to filter out regional differences in consumer taste from overall cost-of-living differences between comparable international and intranational region pairs. Table \ref{tab: border_effects_decomp_taste} presents the results from applying the subclassification estimator to the trimmed sample of region pairs to each of the structural components from equation \ref{eq:cle_decomp2} evaluated at the baseline elasticities of substitution. Because the variance decomposition is exact, the sum of the effects for each of the individual structural components sums to the overall effect, which we replicate in column (1). To obtain the relative importance of LOP deviations and choice set differences, we compare each of their individual effects to the overall effect. We also report the mean of the variance of each of the structural components for intranational pairs in the row indicated by $\mathbb{V}\left(\cdot|l,l^{''} \in \mathcal{L}_{\text{intra}} \right)$.

First, we detect considerable geographic market segmentation across international region pairs. This is because both LOP deviations and choice set differences are significantly higher for international pairs relative to comparable intranational pairs. Taken together, they account for a little over 40\% of the overall cost-of-living differences across international region pairs. Given that overall cost-of-living differences across international regional pairs are seven times larger compared to cost-of-living differences across international region pairs, this result implies a substantial degree of geographic market segmentation across international region pairs. At the same time, intranational region pairs seem to be characterized by a limited amount of geographic market segmentation. Across intranational region pairs, LOP deviations and choice set differences jointly lead to a variance in cost-of-living differences of 0.0105. Therefore, a back-of-the-envelope calculation implies that around 70\% of cost-of-living differences for intranational region falls within a range $[-10.2\%,10.2\%]$ around its mean.\footnote{The back-of-the-envelope calculation we perform is taking the square root of 0.0115 and assuming that this distribution is approximately normal. If so, roughly 70\% of the mass then falls within one standard deviation around a zero mean.}

Second, while both LOP deviations are statistically significantly higher for international region pairs, choice set differences quantitatively dominate LOP deviations as a margin of geographic market segmentation. Out of the 40\% of regional cost-of-living differences attributable to geographic market segmentation, choice set differences explain more than 95\% of the variation. The important role of differences in choice sets to understand cost-of-living differences is qualitatively in line with \citet{Argente2021} and \citet{Cavallo2022} which decompose differences between ICP price levels and cost-of-living differences across countries.\footnote{Our exercise decomposes the variance of level differences in cost of living into different components and therefore our results do not translate one-to-one to these papers.} \citet{Argente2021} find that not accounting for choice set differences can lead to a difference of 25\% between a welfare-based price measure and the ICP measure. \citet{Cavallo2022} measure cost-of-living differences across countries by combining the number of firms and barcodes across countries with a Melitz-Pareto model to estimate the role of differences in variety (broadly defined). They find that up to 36\% of the difference in cost of living relative to a GDP-deflator can be attributed to differences in variety. Relative to these papers that focus on a set of trading partners that still have many formal trade barriers, our results show that differences in choice sets still matter greatly across trading partners that are part of both a currency and customs union. In addition, the fact that choice set differences seem a much more important margin of geographic market segmentation compared to LOP deviations is remarkable in light of the predominant focus on LOP deviations as a measure of geographic market segmentation.

Finally, column (3) confirms that more than 50\% of the overall cost-of-living differences can be accounted for by taste differences. The large role of taste differences is consistent with the evidence provided in \citet{Cosar2018}. They document that home-biased preferences are the main driver of home market advantage in terms of market shares for automobile manufacturers. So, not only do we find that taste differences are key to explaining within-country cost-of-living differences (see Table \ref{tab: border_effects_cle}), but they are also even higher across countries. This has two implications. On the one hand, this implies that even if geographic market segmentation would fall to levels seen for intranational region pairs, large cost-of-living differences across European countries would remain simply because of cross-country differences in consumer tastes. On the other hand, the existence of large taste differences across countries casts doubt on whether only relying on trade shares is very informative about geographic market segmentation. Nevertheless, while both trade shares and cost-of-living differences will yield differences between international and intranational region pairs when consumer tastes vary across countries, our decomposition allows us to separate which part of cost-of-living differences is due to taste differences and which part can be attributed to geographic market segmentation. 

\begin{table}[H]
    \centering
    \caption{Geographic Market Segmentation}
    \label{tab: border_effects_decomp_taste}
    \begin{spacing}{1.1}
        \begin{tabular}{lcccc} \toprule 
            & $\text{CLE}_{ll',t}$ & $\text{LOP}_{ll',t}$ & $\text{Taste}_{ll',t}$ & $\text{Choice}_{ll',t}$  \\
             $\mathbb{\hat{V}}_{ll',t}\left[y_{pl^{'},t}-y_{pl,t}\right]$ & (1) & (2) & (3) & (4) \\ \midrule
            \myinput tables/estimation/P1_3_4_3_NCES_decomp.tex \bottomrule \\
    \end{tabular}
\end{spacing}
    \parbox{\textwidth}{
    \begin{spacing}{1} 
        {\footnotesize 
        \textit{Notes}: This table decomposes the border effect of cost-of-living differences into the structural components. The log of cost-of-living differences is the sum of the structural components. Therefore, it can be subdivided into the variance of each of the structural components and the covariances and we allocate these covariances equally across the components. In this way, the sum of the individual border effects sums to the associated border effect overall for the variance of log cost-of-living differences. We compute each of the components at the baseline elasticities and apply the subclassification estimator at the trimmed sample of region pairs (based on the admissible propensity score range). Column (1) displays the border effects for marginal cost differences and column (2) for markup differences. Columns (3) and (4) display the border effects for expenditure share differences at the product variety level and firm level respectively. Columns (5) and (6) present the border effects for choice set differences at the product variety level and at the firm level. Below the border effects, we display the average variance for the intranational pairs included in the estimation panel and the ratio of the component-specific border effect and the border effect for overall cost-of-living differences. We compute cluster standard errors at the region pairs and present them in brackets below the coefficient estimates. Reported significance levels are at the $p<0.1^{*}$,$p<0.05^{**}$ and $p<0.01^{***}$ levels.}
\end{spacing}}
\end{table} 
