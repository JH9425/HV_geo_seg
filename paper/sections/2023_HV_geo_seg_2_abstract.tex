\begin{abstract}
	\begin{spacing}{1}
        We analyze the degree of geographic market integration in European final goods markets. We propose a unifying framework to analyze geographic market segmentation both in terms of Law of One Price (LOP) deviations and choice set differences. To this end, we decompose regional cost-of-living differences into (1) LOP deviations, (2) pure taste differences and (3) choice set differences. In turn, we detect geographic market segmentation by considering terms whether LOP deviations and choice set differences are larger across international region pairs compared to intranational pairs. Using regionally disaggregate consumption data on 68 fast-moving consumer goods, we estimate that overall cost-of-living differences across Belgium, France, Germany and the Netherlands are more than seven times larger compared to intranational region pairs. Roughly 60\% of the cost-of-living differences are explained by pure taste differences and the other 40\% by the margins of geographic market segmentation. While choice set differences account for 38\% of the variation, LOP deviations explain a mere 2\%. Our results highlight the importance of choice set differences relative to LOP deviations as margins of geographic market segmentation and the presence of large cross-country taste differences across European countries.
    \end{spacing}
\end{abstract}

\begin{spacing}{1}
	\noindent\textbf{JEL codes}: D12, F15 and R32\\
\noindent\textbf{Keywords}: Market segmentation, Cost-of-living, Law of One Price deviations, Taste differences and Choice set variation
\end{spacing}