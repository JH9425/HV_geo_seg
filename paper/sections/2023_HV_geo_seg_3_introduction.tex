\section{Introduction}  

Understanding the presence of geographic market segmentation and the process towards geographically integrated international markets has been a question of central importance to both researchers and policymakers. For instance, the European Single Market (ESM) intends to forge a geographically integrated market from the national goods markets of its member states. A geographically integrated market is a market in which, beyond the costs associated with physically moving the product to the destination market, geography or the nationality of the buyer does not affect the terms at which producers sell their products (see \citet{Flam1992} and \citet{Goldberg1997}). Conversely, if geography or the nationality of the buyer has residual predictive power for the terms at which transactions take place, markets are geographically segmented. Prior work has relied on two strategies to assess whether markets are geographically segmented. One strategy is to investigate whether prices of identical products differ significantly more across countries than across regions of the same country (e.g. \citet{Engel1996} and \citet{Goldberg1997}). Comparing Law of One Price (LOP) deviations across countries to LOP deviations across regions of the same country, \citet{Beck2020} and \citet{Fontaine2020} find considerably higher deviations across countries in Europe. Another strategy is to investigate to what extent trade shares discontinuously fall at national borders (\citet{McCallum1995}). For instance, \citet{Santamaria2023} show that trade shares are substantially lower when goods cross a national European border compared to when they only cut across a regional border.

Conceptually, neither approach is fully satisfactory to measure the presence of goods market segmentation. First, the presence of LOP deviations is not a necessary condition for market segmentation. If fixed market entry costs were the only international trade costs, only the most competitive firms would sell outside their domestic market but it would not necessarily imply that these firms geographically price discriminate.\footnote{In a \citet{Melitz2003} model with no variable trade cost, markets are segmented because choice sets differ, but firms do not price-to-market if consumers are equally price-sensitive.} Thus, even though LOP deviations could be zero, market segmentation still arises through selection into geographic markets.\footnote{In fact, \citet{Cavallo2014} \cite{Beck2020} can only compute LOP deviations for a limited set of observations in their respective datasets.} Second, the presence of a large border effect for trade shares is not a sufficient condition for market segmentation. When consumer tastes are home-biased, trade between international region pairs is lower compared to trade among intranational region pairs regardless of the size of border-related trade costs. This is because if consumer tastes are home-biased, trade between international region pairs is lower compared to trade among intranational region pairs regardless of the size of border-related trade costs.

This paper analyzes geographic market segmentation across final good markets of various European countries (Belgium, France, Germany and the Netherlands) and addresses the aforementioned measurement concerns. To this end, we construct a dataset from samples of households that report their expenditure on all product varieties within 68 fast-moving consumer goods (FMCG) categories. We observe product variety-level prices, quantities, and firm identifiers, all at a fine regionally disaggregate level. We measure geographic market segmentation with a two-step approach. First, borrowing insights from existing theory, we construct cost-of-living differences across all region pairs in our sample and decompose them into (1) LOP deviations, (2) differences in consumer taste and (3) choice set differences. Second, we detect the presence of geographic market segmentation by estimating whether LOP deviations and choice set differences are larger for international region pairs, region pairs separated by a national border, relative to comparable intranational region pairs, which are region pairs within the same country. In this way, our approach addresses the aforementioned concerns by considering both LOP deviations and choice set differences as margins of market segmentation while filtering out differences in consumer taste as a residual source of cost-of-living differences.

Before outlining our two-step approach, we provide reduced-form evidence that both LOP deviations and choice set differences are larger across international region pairs compared to intranational pairs. In particular, the standard deviation of log price differences for identical product varieties within product category-region pair-year cells is, on average, 16\% larger for international pairs relative to intranational pairs. While price differences are considerable, so are choice set differences. At the same time, the share in terms of the number of barcodes that are sold in both regions is on average 75\% for intranational pairs but is below 25\% in most cases for international pairs. Expressing choice set overlap in terms of expenditure or at the firm-level paints a similar picture.

To jointly analyze the role of LOP deviations and choice set differences as margins of geographic goods market segmentation, we specify a structural model of demand that guides the estimation of regional cost-of-living differences. We model preferences as a nested CES utility system, with one nest at the product variety level and the other at the firm level. We choose to work within the family of nested CES demand system as it is the workhorse framework to analyze the benefits from market integration (e.g. \citet{Arkolakis2012}) while providing a good fit for the type of scanner data we work with.\footnote{More specifically, \citet{Dellavigna2019} shows that the implied log quantity-log price relationship fits the price and quantity data well. In addition, \citet{Hottman2016} shows how the nested CES-preference system is flexible enough to account for many empirical patterns that characterize an economy populated by heterogeneous multi-product firms.} We rely on insights from \citet{Redding2020} and obtain, under a normalization of the region-specific geometric average taste level, an expression for regional cost-of-living differences at the product category level that depends on (1) unweighted average LOP deviations, (2) average differences in market share on common varieties and common firms and (3) choice set differences. Importantly, even though average taste levels across regions are normalized, average differences in common market shares still capture mean-preserving differences in consumer tastes. To this expression, we add and subtract a restricted cost-of-living index for common firms and barcodes to which the general expression collapses when taste differences are zero. This allows us to decompose regional cost-of-living differences at the product category level into (1) LOP deviations, now corrected for substitution effects, (2) pure taste differences and (3) choice set differences. 

The importance of taste differences and choice set differences crucially depends on the elasticities of substitution at the product variety and firm level, which we allow to vary across product categories. To estimate the product variety level elasticities, we capitalize on the fact that we also observe at which retail store purchases were made. While stores provide frequent sales and promotions over time, they tend to price uniformly across regions within the same country. After flexibly accounting for seasonal variation in prices and quantities, we instrument product variety level prices with prices of the same product in the same chain but in a different region. We recover a distribution of elasticities of substitution with a 10\%-90\% range of $[-4.77,-1.15]$ across product categories. To estimate the elasticities of substitution at the firm level, we rely on the structure of the nested preference structure. In particular, the firm-level price index can be decomposed into a part that depends on the unweighted geometric average across product-variety level prices within the firm-level nest and a term that captures dispersion in market shares across product varieties within the same firm-level nest. Conditional on a specific fixed effect, the second term is uncorrelated with the firm-level demand shock and we use it to instrument firm-level prices. Doing so, we obtain an estimated distribution of firm-level elasticities of substitution with a range 10\%-90\% range of $[-4.84,-1.71]$.  

With these elasticities in hand, we construct regional cost-of-living differences by first computing the three margins within product category-region pair-year cells and then by calculating the variance of the log cost-of-living differences across product categories within each region pair-year cell. The variance of regional cost-of-living differences is on average 0.12 for intranational pairs. In line with the motivating evidence, the variance of cost-of-living differences for international pairs is roughly 8 times larger compared to the average variance for intranational pairs. To understand the importance of each of the three components in determining the overall cost-of-living differences, we exactly decompose the variance of cost-of-living differences across the three components by distributing the covariance terms equally. Intranational cost-of-living differences are predominantly driven by differences in consumer taste. LOP deviations and choice set differences are quite muted for intranational pairs and explain less than 10\% of the cost-of-living variation. For international regional pairs, LOP deviations, choice set differences and differences in consumer taste are all larger compared to intranational region pairs. Choice set differences and differences in consumer taste explain the bulk of elevated cost-of-living differences for international pairs.

To be able to detect geographic market segmentation from regional cost-of-living differences, we take two more steps. First, the decomposition of cost-of-living differences highlights that, besides LOP deviations and choice set differences, cost-of-living differences also capture pure taste differences. As heterogeneity in preferences is traditionally considered to be outside of the scope of international integration policies, we solely focus on LOP deviations and choice set differences as margins of geographic market segmentation.\footnote{In our view, sustained LOP deviations and choice set differences can occur because of variable trade costs, such as transport costs, border-related costs or differences in distribution margins, and because of fixed market entry costs.} Second, according to the definition of geographic market segmentation, we should only focus on cost-of-living differences that are not related to the costs associated with physically moving goods to different markets. As international region pairs are characterized by greater geographic differences, for instance in terms of physical distance, compared to intranational region pairs, one should compare cost-of-living differences between international and intranational pairs with similar geographic differences.\footnote{The larger distance between international and intranational region pairs can lead to increased cost-of-living differences through transport-induced cost differences (e.g. \citet{Donaldson2018}) or distance-induced selection (e.g. \citet{Hummels2004})} To this end, we follow \citet{Santamaria2021} and construct samples of comparable international and intranational region pairs by estimating the probability that a region pair is an international region pair, and thus separated by a national border, based on geographic determinants.\footnote{With geographic differences we refer to differences in terms of distance, remoteness, elevation differences, shared river basins, and their higher-order interactions.} 

Controlling for geographic differences across region pairs reduces the differences in cost-of-living differences between intranational and international region pairs, but only to a limited extent. Whereas the cost-of-living differences are roughly 8 times larger unconditional on geographic differences, they are a little over 7 times as large conditional on geographic differences.\footnote{As cost-of-living differences intranational region pairs are quite similar conditional and unconditional on geographic differences, the main difference comes from excluding international region pairs which have propensity scores of being separated by a national border close to one.} Above all, we find that a little under 60\% of the increase in cost-of-living differences for international region pairs relative to intranational pairs is explained by pure taste differences. This result implies that trade data alone is unlikely to be very informative about geographic goods market segmentation. Also, a little over 40\% of the increased cost-of-living differences can be attributed to margins associated with market segmentation. Out of the remaining 40\% of the higher cost-of-living differences, 95\% are due to choice set differences and only 5\% of these differences are explained by LOP deviations. 

% \citet{Flam1992} speculates that the bands in which price differences may remain across European markets compared to US markets would be larger because of larger differences in tastes and less labor mobility. In this way, for goods facing higher transport costs larger price differences may remain. While \citet{Santamaria2021,Santamaria2022} show that goods trade (through trucks) is still very nationally segmenteed, \citet{Head2021} find that internal trade barriers among EU-countries are at the level of US states. So, it seems that different datasets and methods have arrived at different conclusions.

We contribute to three strands of literature. Methodologically, our paper is related to the extensive literature on measuring inflation and cost-of-living differences using ideal price indices, using the family of CES preference systems in particular. Building on the work of \citet{Sato1976} and \citet{Vartia1976}, estimating the gains from trade and welfare-relevant inflation figures have made considerable strides by accounting for changes in product variety (e.g. \citet{Feenstra1994} and \citet{Broda2006}) and by incorporating changes in consumer tastes (\citet{Redding2020}), comparatively little is known about how important geographic market segmentation is in explaining spatial cost-of-living differences. \citet{Handbury2015} and \citet{Feenstra2020} measure intranational variation in cost-of-living and \cite{Argente2021} and \citet{Cavallo2022} quantify international variation in cost-of-living levels. However, there is no prior work that combines intra- and international variation which is crucial to understand the role of geographic market segmentation (\citet{Anderson2004}) and this paper fills this gap.

Second, we complement a substantial literature that infers international trade costs from international price differences, or LOP deviations, starting with \citet{Engel1996} and \citet{Goldberg1997}. Moving from aggregate to increasingly detailed price data allowed the literature to overcome measurement issues due to aggregation and confirm the continued presence of large differences on the US-Canada border (see \citet{Broda2008}, \citet{Gorodnichenko2009} and \citet{Gopinath2011}) and across European countries (e.g. \citet{Cavallo2014}, \citet{Fontaine2020} and \citet{Beck2020}). At the same time, focusing on increasingly detailed data also uncovered large underlying choice set differences. To date, there is no study accounting for both LOP deviations and choice set differences as margins of geographic goods market segmentation. We develop a unified framework encompassing both margins and illustrate that focusing on LOP deviations provides a very limited view of the current level of market segmentation across European consumer goods markets. 

Finally, we contribute to a large literature that aims to measure market segmentation by estimating border effects for trade shares. In a seminal contribution, \citet{McCallum1995} established that the US-Canada border led to substantially larger reductions in goods trade than regional borders. Increasingly sophisticated methods were developed to measure national border effects appropriately (\citet{Anderson2003} and \citet{Santamaria2021}), but it is unclear what these border effects imply for geographic market segmentation when heterogeneity in consumer taste is substantial. We measure border effects in terms of cost-of-living differences and separate differences in consumer taste from the two margins that reflect geographic goods market segmentation. We find that variation in consumer taste is substantial which implies that border effects for trade shares only partly reflect geographic goods market segmentation. 

Section \ref{sec:data} introduces the dataset and section \ref{sec:reduced_form} provides motivating evidence for moving beyond LOP deviations when studying market segmentation. Section \ref{sec:theory} introduces the structural framework we use to compute and decompose cost-of-living differences into the three components. Section \ref{sec:struc_est} provides the estimates of the elasticities of substitution. Section \ref{sec:border_effects_eu} estimates which part of the elevated cost-of-living differences can be attributed to geographic market segmentation in goods markets and section \ref{sec:conclusion} provides some concluding remarks. 
